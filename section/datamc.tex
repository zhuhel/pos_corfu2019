The data sets for this analysis were recorded using single and multi-lepton triggers.
The transverse momentum (\pT) thresholds of these triggers vary from 8 to 26~\GeV{},
depending on the lepton flavour and data-taking periods.
The overall trigger efficiency for selected inclusive $ZZjj$ signal events in the analysis region ranges from 95 to 99\%{}.
The dataset corresponds to a luminosity of 139 \ifb after the requirement that the detector is fully functional and the quality of the data is good for physics studies.

The EW $ZZjj$ production is modelled using \MGMCatNLO~2.6.1~\cite{Alwall:2014hca} matrix elements (ME) calculated in the LO approximation
in perturbative QCD (pQCD) and with NNPDF2.3LO~\cite{Ball:2012cx} parton distribution functions (PDF).
The QCD $ZZjj$ production is modelled using \textsc{Sherpa} 2.2.2~\cite{Gleisberg:2008ta} with the NNPDF3.0NNLO~\cite{ball2015parton} PDF,
where events with up to one (three) outgoing partons are generated at NLO (LO) in pQCD.
The production of $ZZjj$ from the gluon-gluon initial state with a four-fermion loop or with an exchange of the Higgs boson has an order of $\alpha_{S}^{4}$ in QCD,
and is not included in the \textsc{Sherpa} simulation.
This process, denoted as the $ggZZjj$ process, is modelled using \textsc{Sherpa} 2.2.2 with the NNPDF3.0NNLO PDF in the \lllljj channel,
and using \textsc{gg2VV}~\cite{Kauer:2013qba} with the CT10NNLO PDF~\cite{Gao:2013xoa} in the \llvvjj channel.
Due to the limited accuracy of ME calculation, the second (both) jets in $ggZZjj$ events are produced in parton showering in the \lllljj (\llvvjj) channel.
The leptonic decays of $Z$ bosons are included in the simulation.
Interference between EW and QCD $ZZjj$ is modelled with \MGMCatNLO~2.6.1 calculated at LO. 

The production of QCD $WWjj$ as well as EW and QCD $WZjj$ with the subsequent leptonic decays of vector bosons are modelled with \textsc{Sherpa} 2.2.2 with NNPDF3.0NNLO PDF.
Diboson processes with the subsequent semileptonic decays ($WW \rightarrow lvqq$ and $WZ \rightarrow qqll$)
are modelled using \textsc{Powheg-Box}~v2~\cite{Frixione:2007nw} with the CT10 PDF~\cite{Lai:2010vv}.
Triboson production is modelled using \textsc{Sherpa} 2.2.2 with NNPDF3.0NNLO PDF.
For top-quark pair production, the \textsc{Powheg-Box}~v2 event generator with the CT10 PDF is used.
The production of single top-quark in $t$-channel, $s$-channel and $Wt$-channel were simulated using the \textsc{Powheg-Box}~v1 event generator~\cite{Alioli:2009je,Frederix:2012dh,Re:2010bp}.
The production of \ttbar~in association with vector bosons ($ttV$) is modelled with \MGMCatNLO~2.3.3 for $ttW$ and $ttZ$ with the $Z$ to $\nu\nu/qq$ decays,
with \MGMCatNLO~2.2.2 for $ttWW$, and with \textsc{Sherpa} 2.2.1 for $ttZ$ with the $Z$ to dilepton decays, respectively.
The \Zjet processes are modelled using \textsc{Sherpa} 2.2.1 with NNPDF3.0NNLO PDF, where the ME is calculated for up to two partons with next-to-leading-order (NLO) accuracy in pQCD and up to four partons with LO accuracy.

The parton showering is modelled with \textsc{Pythia}~8.186~\cite{Sjostrand:2007gs} using NNPDF2.3~\cite{Ball:2012cx} PDF and the A14 set of tuned parameters~\cite{ATL-PHYS-PUB-2014-021}
for all the samples except for those from \textsc{Sherpa}, where parton showering is simulated within the \textsc{Sherpa} programme.

All simulated events were processed with a detailed detector simulation~\cite{SOFT-2010-01} based on \textsc{Geant4}~\cite{Agostinelli:2002hh}.
Furthermore, simulated inelastic $pp$ collisions were overlaid to model additional $pp$ collisions in the same and neighbouring bunch crossings~(pileup).
Simulated events were reweighted to match the pileup conditions in the data. All simulated events were processed using the same reconstruction algorithms as used in data.
Furthermore, the lepton and jet momentum scale and resolution, and the lepton reconstruction, identification, isolation and trigger efficiencies in the simulation were corrected to match those measured in data.


