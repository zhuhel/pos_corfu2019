This analysis performs cross-section measurements in the fiducial volumes as well as a statistical fit to MD distributions to extract the EW $ZZjj$ contributions. Therefore, experimental and theoretical uncertainties may influence the analysis in the predictions of background yields and MD shapes,
correction factors to extrapolate the QCD and EW $ZZjj$ events from detector-level to fiducial volume
(i.e. $C$-factors, calculated as the ratio of the number of $ZZjj$ events passing the detector-level event selection to the number of events selected in the fiducial volume), as well as $ZZjj$ MD shapes.
The statistical uncertainties of the simulated samples for both the signal and background processes are also taken into account.
The systematic uncertainty sources that affect $ZZjj$ production are detailed below.
%, while the uncertainties specific to background estimation will be discussed in the following section.

The major experimental uncertainties originate from the luminosity uncertainty, the momentum scale and resolution of leptons and jets,
as well as from the lepton reconstruction and selection efficiencies.
Smaller experimental uncertainties are also considered, such as those due to the trigger selection efficiency, the calculation of the \met~soft-term, the pile-up correction, and the $b$-jet identification efficiency. Overall, the total experimental uncertainty in $C$-factors~is 5--10\%, dominated by the jet and lepton components. The uncertainty in the combined 2015--2018 integrated luminosity is 1.7\%~\cite{ATLAS-CONF-2019-021}, obtained using the LUCID-2 detector \cite{LUCID2} for the primary luminosity measurements.
In addition, a conservative uncertainty is assigned on the QCD $ZZjj$ processes by comparing the MD distributions in low and high pile-up conditions, to account for a potential mismodelling of pile-up in simulation.

The theoretical uncertainties on the EW and QCD $ZZjj$ processes include the uncertainties from PDF, QCD scales, $\alpha_{S}$, and parton showering. The PDF uncertainty is estimated following the PDF4LHC~\cite{Butterworth:2015oua} procedure, where the envelope of the NNPDF internal errors and the differences between the nominal and alternative PDFs are considered as the final uncertainty.
The QCD scale uncertainty is estimated by varying independently by factors of 0.5 to 2.0 the nominal renormalisation and factorisation scales ($\mu_r$ and $\mu_f$),
which results in seven different configurations excluding the two extreme variations, ($\mu_r = 2$, $\mu_f = 0.5$) and ($\mu_r = 0.5$, $\mu_f = 2$),
where the largest deviation is chosen as the uncertainty.
The parton showering uncertainty is estimated by comparing the nominal \textsc{Pythia8} parton showering with the alternative \textsc{Herwig7}~\cite{Bellm:2015jjp, Bahr:2008pv} algorithm. The $\alpha_{S}$ uncertainty is estimated by varying the $\alpha_{S}$ value within $\pm$ 0.001. The interference effect between the EW and QCD processes is checked with \MGMCatNLO~2.6.1 at particle level, and found to be +7(+2)\% of the EW contribution in the fiducial volume in the \lllljj (\llvvjj) channel. This effect is taken as an additional uncertainty in the EW $ZZjj$ predictions. The total theoretical uncertainty in the fiducial volume yields for the EW (QCD) $ZZjj$ process is estimated to be about 10\% (30\%), where the large uncertainty in the QCD prediction is dominated by the QCD scale uncertainty.
As the shape of QCD $ZZjj$ production is critical in the determination of EW $ZZjj$ signal contributions,
an additional uncertainty affecting the MD shapes (`generator modelling uncertainty') is considered,
estimated by comparing \textsc{Sherpa} with \MGMCatNLO~2.6.1 predictions at particle level, where two partons are explicitly required in the ME calculation.

